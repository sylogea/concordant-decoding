\documentclass[11pt]{article}

\usepackage[preprint]{acl}
\usepackage[T1]{fontenc}
\usepackage[utf8]{inputenc}
\usepackage{graphicx,inconsolata,latexsym,microtype,times}

\usepackage{amsfonts,amssymb,bm,booktabs,cancel,caption,centernot,dsfont,graphicx,marvosym,mathtools,nicefrac,physics,stackengine,subcaption,textcomp,tikz,xcolor,xparse}
\usepackage[scr=boondox]{mathalpha}

\usepackage{algorithm,algorithmicx,algpseudocode}
\algrenewcomment[1]{\textcolor{black!50}{\small\text{\# }\textit{#1}}}

\usepackage{amsthm}
\theoremstyle{definition}
\newtheorem{definition}{Definition}
\newtheorem{notation}{Notation}
\newtheorem{theorem}{theorem}
\newtheorem{proposition}{Proposition}
\newtheorem{lemma}{Lemma}
\newtheorem{claim}{claim}
\newtheorem{remark}{Remark}

\usepackage{parskip}
\makeatletter
\newlength{\thmextra}
\setlength{\thmextra}{\parskip}
\addtolength{\thmextra}{2pt}
\def\thm@space@setup{\thm@preskip=\thmextra \thm@postskip=0pt}
\makeatother

\DeclareMathOperator*{\argmax}{arg\,max}
\DeclareMathOperator*{\argmin}{arg\,min}

\newcommand{\quotes}[1]{\text{``#1''}}
\newcommand{\N}{\mathbb{N}}
\newcommand{\R}{\mathbb{R}}
\newcommand{\inner}[1]{\langle #1\rangle}
\newcommand{\eostoken}{\#}
\newcommand{\mdoubleplus}{\mathbin{+\mkern-11mu+}}

\title{Concordant Decoding}

\author{
	Gregory Lim \and Second Author \and Third Author \\
	Singapore University of Technology and Design \\
	\texttt{\{gregory\_lim, ...\}@mymail.sutd.edu.sg}
}

\begin{document}

\maketitle

\begin{abstract}
	Main contributions and findings (200 words).
\end{abstract}

\section{Introduction}
Blah.

\section{Related Work}
Blah.

\section{Methodology}
% Introduce the rule
Let $V$ be a finite vocabulary totally ordered by $\preceq$,
and $\Pr$ be a next-token kernel of an autoregressive language model.
Let $\ell$ denote $\log\Pr$,
and juxtaposition denote concatenation.

We define the decoding rule by
\begin{equation}
	\argmax_{y\in V}\left[
		\ell(y\mid x)
		+
		D(p\,\Vert\,q)
	\right]
\end{equation}
where $D$ is KL-divergence,
and
\[
	p
	=
	\{
		\Pr(z\mid xy)
	\}_{z\in V},
\]
and
\[
	q
	=
	\{
		\Pr(z\mid x)
	\}_{z\in V}.
\]

% Introduce the approximate algorithm

% and its computational complexities
% Fix a uniform-cost random-access machine model of computation.

\section{Experimentation}

\section{Conclusion}

\bibliography{main}

\appendix
\section{Appendix Title}
Appendix content goes here \citep{example2025}.

\end{document}
